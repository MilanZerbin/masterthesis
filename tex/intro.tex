\subsection*{History and Motivation}
The illustrious theorem of Gauss--Bonnet in its historic form relates  the Gauss-cuvature $K$ on an embedded surface $S\subset \R^3$ to its Euler-characteristic $\chi$ via
\begin{equation*}
\int_S K \,\mathrm{d}A = 2\pi\,\chi(S).
\end{equation*}
As an immediate consequence, the only surface that may be embedded into $\R^3$ with everywhere positive curvature is $S^2$, for if the genus $b$ of a surface is positive, the right side of the equation will be nonpositive as $\chi(S) = 2 - 2b$.
Nowadays the theorem of Gauss-Bonnet is often held to be the first incidence of a series of striking results coupling curvature restrictions and topology.\\
Let us introduce some terms to make the above connection more precice: 
Given a riemannian metric $g$ on a manifold $M$, one higherdimensional analogue of the Gauss-curvature is the scalar curvature $\scal_g\colon M\to \R$.
We call $g$ a \psc-metric if its scalar curvature is everywhere positive.
Given any manifold $M$, the first question to ask in the spirit of the previous is whether $M$ admits a \psc-metric, the second then concerns uniqueness of such a metric.
Since scaling a metric by a factor simply rescales its scalar curvature, the uniqueness question is foolish.
If we can deform one \psc-metric smoothly to every other such metric, keeping scalar curvature positive along the way, we might still consider the metric to be unique in some sense.
Therefore one should more accurately ask:
\begin{q}
    Can we understand the homotopy type of the space of \psc-metrics on a given manifold?
\end{q}\noindent
For future reference we name this space of metrics with positive scalar curvature $\mR^+(M)$, existence and uniqueness in the above sense are thus encoded in $\pi_0(\mR^+(M))$. 
This space turns out to sometimes have very intricate and complicated topology, see \cite{hanke:psc} for some indications.
Exemplary shown in \cite{erw:psccc}, already the connected component of the round metric in $\mR^+(S^d)$ for $d\geq 6$ has the rich topology of an infinite loop space.
The question of existence for simply connected manifolds was however almost completely answered independently in \cite{gl:scpsc} and \cite{sy:scpsc} by the following:
\begin{theorem*}[Gromov--Lawson, Schoen--Yau, 1980]
    Let $M$ be a simply connected manifold of dimension $n \geq 5$. If $M$ is not spinnable, it admits a \psc-Metric. Furthermore for spinnable $M$, a neccesary condition for the existence of such a metric is given by the vanishing of the $\hat{\mathfrak{a}}$-genus of $M$.
\end{theorem*}
Approximately twelve years later, Stolz gave a proof \cite{sto:suf} that the condition in the spinnable case is also sufficient.
Later, the central ingredient in the proof of the theorem above was stengthened in \cite{cher:sur} and \cite{kord:sur} to more general results allowing some comparison of $\mR^+(M)$ with $\mR^+(M^\prime)$ when $M^\prime$ is built from $M$ via specific surgeries, see \cite{georg:diss} for a selfcontained introduction and proof.
The further development of these surgery results leads to the main source and inspiration for this thesis, the paper \cite{ew:psc}, in which they prove:
\begin{theorem*}[Ebert--Wiemeler, 2022]
    Let $M$ and $M^\prime$ be $d$-manifolds of tangential $2$-type $(B,\t)$ and $x\in \Omega_d^\t$ a bordism class containing a representative with \emph{admissible splitting}, such that $[M] \equiv [M^\prime] + x$ in the $\t$-bordism group $\Omega_d^\t$. 
    Then there is a weak homotopy equivalence
    \begin{equation*}
        \mR^+(M) \whe \mR^+(M^\prime)
    \end{equation*}
    provided that $d\geq 5$.
\end{theorem*}
They illustrate the use of their theorem by setting $B = \BSpin$ and proving that for any simply connected spin-manifold $M$ of dimension $d\geq 5$ one has $\mR^+(M) \whe \mR^+(S^d)$ if $M$ admits a $\psc$-metric.
Essentially, taking their theorem for granted such proofs now only consists of studying the structure of the involved bordism ring $\Omega^\t$, which is usually well understood, and having an ample supply of manifolds which admit an admissible splitting.
Fortunately, they give a plentitude of manifolds which admit such an admissible splitting in the paper, making it rather simple to reproduce similiar results to theirs for other tangential types $\t$.
They also take care of the case $B = \BSO$ themselves, and remark:
\begin{displayquote}\itshape
    Using the methods of this paper and computations in cobordism theory, one can prove partial results for many other [tangential] $2$-types, such as $\BO$, $\BO\times\B G$ or $\BSO\times\B G$. We refrain from stating them.
\end{displayquote}
This thesis proves such a partial result for the tangential $2$-type $(\BO,\id)$ using exactly these methods. 
\pagebreak{}

\section*{Structure}
A supplementary goal of this text is to give a detailed yet concise introduction to the language of tangential structures, which is hard to find in the literature up to now.
This introduction is given in \hyperlink{section.1}{chapter 1 of this work}, culminating in a classification of all possible tangential $2$-types into six families.
We will find, that a manifold has tangential type $\id\colon\BO \to \BO$, if it is nonorientable, and neither the manifold nor its universal cover admit a spin structure.
\hyperref[sec:bordism]{Chapter 2} introduces the bordism group $\Omega_\ast^\t$ for a general tangential structure $\t$, and finishes with an in depth study of the unoriented bordism ring $\NN_\ast = \Omega_\ast^\BO$.
\hyperref[sec:psc]{The last chapter} introduces the reader very briefly to the world of scalar curvature by gathering the neccesary ingredients to apply the theorem from \cite{ew:psc}.
Then in the \hyperlink{subsection.3.2}{final section of the last chapter}, all comes together to prove the main theorem of this thesis:
\begin{theorem*}
    Let $d\in\N$ be odd or $d>10$. Let $M,M^\prime$ be $d$ dimensional nonorientable manifolds with fundamental group $\zz$ and nonspinnable universal cover. Then $\mR^+(M)\whe \mR^+(M^\prime)$.
\end{theorem*}
We also state and prove according theorems for $d=6,8,10$, where there are up to $4$ possible homotopy types of $\mR^+(M)$. 
In dimensions less than $4$ we show that there are no manifolds of tangential type $\id_\BO$.
There are no results in dimension $4$.
The computation giving the desired result in the end consists of only very elementary considerations and may be appreciated by a wide audience, as the technical ingredients can be black-boxed at will.
In a nutshell: The magic of the machinery built by the ones before us makes it possible to prove theorems about positive scalar curvature without having the slightest idea about scalar curvature.
