\subsection{The main result}
We specialize to the case $\t= \id\colon \BO \to \BO$. 
Every manifold is then a canonically a $\t$-manifold.
By our classification result \ref{tttypes}, a manifold $M$ has tangential $2$-type $\id_\BO$, which is in the first type family, if and only if it satisfies
\begin{enumerate}[label=(\alph*), noitemsep]
    \item The manifold has fundamental group $\pi_1(M) \cong \zz$.
    \item The manifold is nonorientable by $w_1(M)\neq 0$.
    \item The universal cover of $M$ is nonspinnable by $w_2(\widetilde{M}) \neq 0$.
\end{enumerate}
Let us fix a dimension $d$.
Suppose that $\mathcal{E}_d$ is some set generating additively the unoriented bordism group $\NN_d$.
Define
\begin{equation*}
    \mathcal{E}_d^{\text{good}} = \{ x \in \mathcal{E}_d \:\vert\: x \text{ contains a representative with admissible splitting}\}
\end{equation*}
And call $\mathcal{A}_d$ the subgroup of $\NN_d$ generated by $\mathcal{E}_d^\text{good}$.
Then we may apply the theorem from \cite{ew:psc} to obtain the following 
\begin{blueprint*}
    \color{themecolordark}
    Two $d$-dimensional manifolds $M,M^\prime$ satisfying (a),(b),(c) with $[M] \equiv [M^\prime]$ in $\NN_d / \mathcal{A}_d$ have
    \begin{equation*}
        \mR^+(M) \whe \mR^+(M^\prime)
    \end{equation*}
    provided that $d\geq 5$.
\end{blueprint*}
\prf
Notice that the elements in $\NN_d$ which contain a representative with an admissible splitting form a subgroup, since admissible splittings may be constructed componentwise in a disjoint union.
Thus $[M] \equiv [M^\prime]$ in $\NN_d / \mathcal{A}_d$ implies there is a $x\in\NN_d$ with a representative which admits an admissible splitting and $[M] = [M^\prime] + x$.
Then the theorem of Ebert--Wiemeler applies.
\endprf
All there is left to to, is to find as many manifolds with admissible splitting in our list $\mathcal{E}$ from \ref{eprop} as possible.
Thankfully, Ebert and Wiemeler provide a multitude of such manifolds in section 3 of their paper.
We will first use the following result of theirs to show that most of our generators have an admissible splitting.
Shortly after, we will then spruce up the result with \ref{diskbdlsplit}, to obtain even more admissible splittings.
\begin{thesislemma}[Ebert--Wiemeler, thm 3.1 in \cite{ew:psc}]
    Let $M \to M^\prime$ be a smooth fiber bundle with fiber $\KP^{n + m - 1}$. Then $M$ has an admissible splitting if either ...
    \begin{enumerate}[label=$\dots$, noitemsep]
        \item $\K = \R$, $n,m\geq 3$ and the bundle has structure group $O(m)\times O(n) / \{\pm 1\}$
        \item $\K = \C$, $n,m \geq 2$ and the bundle has structure group $U(n)\times U(m)$.
        \item $\K = \C$, $n + m -1 \geq 3$ with structure group $\zz$ acting by complex conjugation.
        \item $\K = \HH$, $n,m\geq 1$ and the bundle has structure group $\mathrm{Sp}(n)\times\mathrm{Sp(m)}/\{\pm 1\}$
    \end{enumerate}
\end{thesislemma}
As a direct corollary, any product $\RP^d\times M^\prime$ with $d \geq 5$ has an admissible splitting, as a trivial bundle admits a structure group reduction to the trivial group, and thus fulfills any such requirement.
By the same argument, any product $\CP^d\times M^\prime$ with $d\geq 3$ also admits admissible splitting, and any product $\HP^d \times M^\prime$ with $d\geq 1$ similarly. 
Furthermore, the Dold-manifold $P(m,n)$ is the total space of smooth a fiber bundle 
\begin{equation*}
    \CP^n \fib P(m,n) \overset{p}{\longrightarrow} \RP^m
\end{equation*}
with $p([x,z]) = [x]$. 
The transition maps perform complex conjugation, so the structure group of this fiber bundle is $\zz$, see \cite{dold:bord}.
For $n\geq 3$ therefore also $P(m,n)$ has an admissible splitting, and any product $P(m,n)\times M^\prime$ hence too.
To summarize,
any element in $\mathcal{E}$ with a representative accomodationg a factor of a suitably high dimensional projective space or Dold-manifold will have an admissible splitting.
This already includes most elements. In fact:
\begin{thesisprop}\label{fbo}
    All elements in $\mathcal{E}$ contain a representative with admissible splitting, except maybe those represented by a manifold of form
    \begin{equation*}
        (\CP^2)^{o_4} \times (\RP^2)^{o_1} \times (\RP^4)^{o_2} \times P(1,2)^k
    \end{equation*}
    with $o_i \in \{0,1\}$ and $k\in \N_0$.
\end{thesisprop}
\prf
By \ref{eprop} a general element in $\mathcal{E}$ is represented by 
\begin{equation*}
    M = \prod\limits_{i\in I} (\CP^{2i})^{n_i} \times \prod\limits_{j\in J} (\RP^{2j})^{o_j} \times \prod\limits_{k\in K}{P(2^{r_k} - 1,s_k2^{r_k})}
\end{equation*}
    with $I,J,K$ finite subsets of the natural numbers, $n_i,r_k,s_k$ positive natural numbers and $o_j \in \{0,1\}$.
    If $n_i$ is nonzero for some $i\geq 2$, we have a factor of $\CP^d$ with $d\geq 4$, so $M$ has an admissible splitting.
    If $o_j$ is nonzero for some $i\geq 3$, we have a factor of $\RP^d$ with $d\geq 6$, so $M$ has an admissible splitting.
    Since $s_k 2^{r_k} < 3$ only for $s_j = r_j = 1$, the only Dold-manifold whose appearance does not a priori make the product have an admissible splitting is $P(1,2)$.
    We are thus left with manifolds of form
\begin{equation*}
    M = (\CP^2)^{n_1} \times (\RP^2)^{o_1} \times (\RP^4)^{o_2} \times P(1,2)^{n_2}
\end{equation*}
    with $n_i\in \N_0$ and $o_i\in\{0,1\}$.
    Using Rokhlin's-trick, if $n_1\geq 2$ we obtain a factor $\HP^2$ from $\CP^2\times\CP^2$, and the product again has an admissible splitting.
    Thus it suffices to consider $n_1\in \{0,1\}$.
\endprf
%Since $P(1,2)$ is fivedimensional, $P(1,2)^k$ has dimension $5k> 6$ if $k \geq 2$.
%Furthermore $P(1,2)$ is orientable, hence by \ref{prodlem} any product $P(1,2)^2 \times M^\prime$ with nontrivial $M^\prime$ is bordant to a manifold with an admissible splitting.
%We may thus reduce the list of outliers further.
%\begin{thesisprop}[Refinement of \ref{fbo}]\label{plo}
    %All elements in $\mathcal{E}$ contain a representative with admissible splitting, except maybe those represented by 
    %\begin{equation*}
        %\begin{split}
        %\RP^2, \CP^2, \RP^4, \RP^4\times\RP^2, \CP^2\times\RP^2, \CP^2\times\RP^4, \CP^2\times\RP^4\times\RP^2, P(1,2),  \\
           %P(1,2)^2, P(1,2)\times\RP^2, P(1,2)\times \CP^2, P(1,2)\times\RP^4, P(1,2)\times\RP^4\times\RP^2
        %\end{split}
    %\end{equation*}
    %In particular all elements of dimension $d>11$ contain a representative with admissible splitting.
%\end{thesisprop}
%\prf
%As both $P(1,2)\times P(1,2)$ and $P(1,2)\times\CP^2$ are orientable manifolds of dimension larger then six, \ref{prodlem} applies to any product of form $P(1,2)^2\times M^\prime$ and $P(1,2)\times \CP^2 \times M^\prime$ with nontrivial $M^\prime$.
%This refines \ref{fbo} to the given list.
%\endprf
%This leaves only finitely many outliers, which is a marvellous start, but further refinement remains possible.
Let us refine the result further.
Not only do Ebert and Wiemeler prove that $[P(1,2)]$ has a representative with an admissible splitting, they give an explicit construction of a certain $5$-manifold which satisfies:
\begin{thesislemma}[Ebert--Wiemeler, prop.~5.8 in \cite{ew:psc}]\label{pundoes}
    There is a five manifold $W$ which has an admissible splitting $W=W_0 \cup W_1$ with $W_i$ two $D^3$-bundles over $S^2$ such that $[W]\neq 0$ in the oriented bordism group $\Omega_5^{\BSO\to\BO}$.
\end{thesislemma}
Already Wall computed in \cite{wall:bord} that $\Omega_5^{\BSO\to\BO}  = \zz$ and the orientation forgetting map from $\Omega_5^{\BSO\to\BO}$ to $\NN_5$ is an isomorphism.
Hence the manifold $W$ from the theorem satisfies $[W] \equiv [P(1,2)]$ in $\NN_5$.
Of course this disk bundle decomposition is precicely what is needed for an application of \ref{diskbdlsplit}.
Taking the representative $W\times M^\prime$ for products of form $[P(1,2)\times M^\prime]$, we have an representative that has an admissible splitting, when $M^\prime$ has a \psc-metric.
Iterated application results in:
\begin{thesisprop}[Refinement of \ref{fbo}]
    All elements in $\mathcal{E}$ contain a representative with admissible splitting, except maybe those represented by 
    \begin{equation*}
        \RP^2, \CP^2, \RP^4, \RP^4\times\RP^2, \CP^2\times\RP^2, \CP^2\times\RP^4, \CP^2\times\RP^4\times\RP^2.
    \end{equation*}
    In particular all elements of dimension $d>10$ contain a representative with admissible splitting.
\end{thesisprop}
\prf
The Dold-manifold $P(1,2)$ has a representative $W$ with admissible splitting by \ref{pundoes}, so it drops from the list of outliers in \ref{fbo}.
This representative also admits a \psc-metric by our remark after the definition of admissible splittings,
so we may apply \ref{diskbdlsplit} also to pure products $P(1,2)^k$, removing them from the list.
By \ref{prodpsc} any product $\RP^k\times M^\prime$ admits a \psc-metric, so \ref{diskbdlsplit} applies to products $P(1,2)\times\RP^k\times M^\prime$ , and those drop from the list in \ref{fbo}.
By representing $\CP^2$ as $\RP^2\times\RP^2$ with the same argument $[P(1,2)\times\CP^2\times M^\prime]$ has a represenative with an admissible splitting (it can of course also be shown more directly that $\CP^2$ admits a \psc-metric, leading to the same result).
We hence have removed all occurences of $P(1,2)$ from the list of outliers, giving the desired result.
\endprf
There is an alternative proof that $P(1,2)$ admits a \psc-metric utilizing a theorem from Gromov--Lawson:
The fibration $\CP^2 \fib P(1,2) \to S^1 = \RP^1$ gives the LES
\begin{equation*}
    0 = \pi_1(\CP^2) \to \pi_1(P(1,2)) \overset{p_\ast}{\longrightarrow} \pi_1(S^1) \to \pi_0(\CP^2) = 0
\end{equation*}
So we get a generator $p_\ast^{-1}(\id_S^1) \in \pi_1(P(1,2)) \cong \Z$.
This generator is represented by an embedding $S^1\to P(1,2)$ as $2\dim(S^1) + 1 = 3 < 5 = \dim(P(1,2))$. 
Since $P(1,2)$ is orientable, this embedding has trivial normal bundle, so we may thicken it to an embedding $S^1\times D^{4} \to P(1,2)$.
Performing a dimension $2$ surgery move we replace this $S^1\times D^{4}$ by a $D^2 \times S^3$, killing the generator of $\pi_1(P(1,2))$ in the process.
We obtain a simply connected five dimensional manifold, which by \cite{gl:scpsc} admits a \psc-metric.
As $5-2 = 3 \geq 3$, the codimension of the surgery was big enough, so that we may conclude $P(1,2)$ admits a \psc-metric aswell.
This killing of $\pi_1$ marks just one incidence of a general result obtainable with surgery theory. 
Any manifold whose tangent bundle trivializes over the $k$-skeleton can be made $k$-connected with surgery moves, see \cite{milnor:surg}.\\
Let us continue towars the main theorems.
In the notation of the {\color{themecolordark} blueprint} the proposition translates to $\mathcal{E}_d^{\text{good}} = \mathcal{E}_d\setminus \mathcal{E}_d^{\text{bad}}$ where 
\begin{equation*}
    \begin{split}
        \mathcal{E}_6^\text{bad} &= \{\CP^2\times\RP^2, \RP^4\times\RP^2\}\\
        \mathcal{E}_8^\text{bad} &= \{\CP^2\times\RP^4\}\\
        \mathcal{E}_{10}^\text{bad} &= \{\CP^2\times\RP^4\times \RP^2\}
    \end{split}
\end{equation*}
and empty $\mathcal{E}_d^\text{bad}$ for $d\geq 5$ else. 
We omit of course the dimesions $1,2,3,4$ where our theorems have no grip.
With a swift look at \fref{fig:boboring} containing our table of the additive generators of the lowdimensional groups $\NN_d$, we compute the quotient $\NN_d/\mathcal{A}_d$ to be $\{S^d\}$ for $d$ odd or $d>10$, and 
\begin{equation*}
    \begin{split}
        \NN_2/\mathcal{A}_6 &= \{S^6, \CP^2\times\RP^2, \RP^4\times\RP^2, (\CP^2\times\RP^2) \connsum (\RP^4\times\RP^2)\}\\
        \NN_2/\mathcal{A}_8 &= \{S^8, \CP^2\times\RP^4\}\\
        \NN_2/\mathcal{A}_{10} &= \{S^{10}, \CP^2\times\RP^4\times\RP^2\}
    \end{split}
\end{equation*}
At last, we use our {\color{themecolordark} blueprint}. 
We obtain three almost identical theorems for the exceptional dimensions $d=6,8,10$, and one general theorem for all other dimensions $d\geq 5$.
In dimension $1,2$ and $3$, the universal cover of any manifold is trivially spinnable as discussed, so there are no manifolds of tangential $2$-type $\id_\BO$ in these dimensions. 
Hence vacuously any (nonexistant) such manifold of odd dimension $d< 5$ satisfies whatever we may wish.
Applying our {\color{themecolordark} blueprint} to all the cases where $\NN_d/\mathcal{A}_d$ is the trivial group yields:
\begin{theorem}[Large or odd dimensions]
    Let $d\in \N$ be odd or $d>10$. Let $M,M^\prime$ be any two $d$-dimensional manifolds of tangential $2$-type $\id_\BO$. Then $\mR^+(M) \whe \mR^+(M^\prime)$.
\end{theorem}
For even $d\geq 6$ we know Dold--manifolds of tangential $2$-type $\id_\BO$, namely write $d= 4k + 3 + (-1)^k$ for a unique $k\in\N$, then $P(3 + (-1)^k,2k)$ is a Dold-manifold of type $\id_\BO$ in dimension $d$ by our table in \fref{fig:pmn}. 
Alternatively, manifolds of form $S^{d-6}\times\CP^2\times\RP^2$ of course give examples of nullbordant manifolds satisfying (a),(b),(c) in dimensions $d\geq 8$ or $d=6$.
We write our theorem more concretely using these manifolds and the transitivity of $\whe$:
\begin{theorem}[Explicit version of theorem 1]
    Let $d\in\N$ satisfy $d=9$ or $d> 10$, and $M$ be a $d$-dimensional manifold of tangential $2$-type $\id_\BO$. Then
    \begin{equation*}
        \mR^+(M) \whe \mR^+(S^{d-6}\times \CP^2\times\RP^2)
    \end{equation*}
\end{theorem}
\prf
Since in the dimensions of the theorem $\NN_d/\mathcal{A}_d = \{0\}$, any manifold of tangential $2$-type $\id_\BO$ is bordant to a given representative, only adding manifolds which admit an admissible splitting.
$S^{d-6}\times\CP^2\times\RP^2$ has fundamental group $\zz$ and is nonorientable because of the $\RP^2$ factor. 
Its universal cover is given by $S^{d-6}\times\CP^2\times S^2$, a manifold with nontrivial $w_2$ by the Whitney product theorem.
Therefore $S^{d-6}\times\CP^2\times\RP^2$ is such a manifold of tangential $2$-type $\id_\BO$.
\endprf
One exemplary application of the theorem is the conclusion $\mR^+(P(4,4)) \whe \mR^+(S^6\times\CP^2\times\RP^2)$.
We remark, that indeed $P(4,4)$ is not nullbordant, as already its Stiefel--Whitney-number corresponding to the partition $(1,1,1,1,1,1,1,1,1,1,1,1)$ does not vanish,
hence a statement like this could not have been proven by a Chernysh-like theorem.\\
In dimension seven, $\NN_7/\mathcal{A}_7$ is trivial aswell, it just does not fit into above theorem as $S^1\times\CP^2\times\RP^2$ has the wrong fundamental group to be a manifold of $\id_\BO$ 2-type.
If one finds a nice seven dimensional manifold of the right tangential $2$-type, one may write down a similar theorem in dimension $7$.
Certainly such manifolds can be found, but we have not found a  good representative to write down a theorem.
Let us finish our analysis in the case of dimension eight.
\begin{theorem}[Dimension $8$]
    Let $M$ be any $8$-dimensional manifold of tangential $2$-type $\id_\BO$. Then we have one of the following:
    \begin{equation*}
        \mR^+(M) \whe \mR^+(S^2\times\CP^2\times\RP^2), \text{ or } \mR^+(\CP^2\times\RP^4)
    \end{equation*}
\end{theorem}
\prf
We just need to show that $\CP^2\times\RP^4$ indeed satisfies (a),(b),(c), as we already know it is not nullbordant mod $\mathcal{A}_8$.
As $\RP^4$ is nonorientable, so is the product, and its universal over $\CP^2\times S^4$ is not spinnable since the $\CP^2$-factor provides a nonzero second Stiefel--Whitney-class.
Furthermore, the fundamental group of $\CP^2$ is trivial so indeed we have $\pi_1(\CP^2\times\RP^4) \cong \zz$.
\endprf
In dimension $10$, we need a manifold $W$ of tangential $2$-type $\id_\BO$ with $[W]\not\equiv 0$ in $\N_{10}/A_{10}$ to formulate the analogous theorem:
\begin{theorem}[Dimension $10$]
    Let $M$ be any $10$-dimensional manifold of tangential $2$-type $\id_\BO$. Then we have one of the following:
    \begin{equation*}
        \mR^+(M) \whe \mR^+(S^4\times\CP^2\times\RP^2), \text{ or } \mR^+(M)
    \end{equation*}
\end{theorem}
Let us sketch how some difficulties we encountered searching for such a $M$. 
By surgery techniques, one may only kill (parts of) the fundamental group if $M$ is orientable.
So let us start with an $10$ dimensional nonorientable manifold $M$ with fundamental group $\zz$, such as $\RP^{10}$.
If $S$ is a simply connected manifold, we have $\pi_1(M\connsum S) \cong \pi_1(M)$ by Seifert--van-Kampen (here we need dimension bigger than $2$).
By a simple Mayer--Vietoris argument one shows $w(M\connsum S) = w(M) + w(S) - 1$.
Since $S$ is orientable, we thus must not fear trivializing $w_1$ in $M\connsum S$, and we may make $w_2$ nontrivial by choosing suitable $S$.
The universal double cover of $M\connsum S$ is built from the double cover $\widetilde{M}$ by attaching two copies of $S$, so $w_2(\widetilde{M}) = w_2(M)$, and we made no progress.
Dimension six is even more difficult, as there are four possible cases, in each of which we would need a representative of the correct $2$-type. 
The $\CP^2\times\RP^2$ case already has such representatives by the above arguments, but for the others an ad-hoc choice would have to be made.
We do not bother to continue the search for such manifolds, and leave the discussion as is.\\
It would be of course interesting to complete the picture by arguing wheter the given cases are further reducible.
We could for example ask questions like
\begin{q}
    Is $\mR^+(S^2\times\CP^2\times\RP^2) \whe \mR^+(\CP^2\times\RP^4)$?
\end{q}\noindent{}
Similarly of course it would be most interesting to complete the picture in dimension $4$, but the author is not aware of any methods applying to those questions.
