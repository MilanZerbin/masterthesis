\subsection{Tangential type of a manifold}
The last section discussed questions of the form: Given a tangential structure $\theta$, which manifolds carry a $\theta$-structure?
This section takes a glance in the opposite direction.
What is the \emph{most tangential structure} we can equip a given manifold with? 
This is answered by the following lesser known invariant.
\begin{defi}[Tangential $k$-type]
    Let $k$ be any positive integer, $(B,\t)$ a tangential structure and $M$ a $\t$-manifold. If both ...
    \begin{itemize}[noitemsep, label=$\dots$]
        \item $\t \colon B\to \BO$ is $k$-coconnected
        \item the lift of $M \to B$ of $\st_M$ is $k$-connected
    \end{itemize}
    the tangential structure $(B,\t)$ is called tangential $k$-type of $M$.
\end{defi}
We will denote the $k$-th stage in the \emph{Moore-Postnikov--factorization} of $\st\colon M \to \BO$ by 
\begin{center}
    \begin{tikzcd}[row sep=small]
        & B^k(M) \arrow[dr, "\t"] & \\
        M \arrow[rr, "\st"] \arrow[ur, "\st^k"] & & \BO. 
    \end{tikzcd}
\end{center}
Now $\t$ is indeed $k$-coconnected, a fibration, and $\st^k$ is $k$-connected, so we have succeeded in our attempt to prove existence of the tangential $k$-type. 
The next lemma concerns uniqueness.
\begin{thesislemma}\label{ttunique}
    If $(B,\t)$ and $(B^\prime,\t^\prime)$ are two tangential $k$-types of a manifold $M$, then there is a homotopy equivalence of fibrations $\t\whe\t^\prime$.
\end{thesislemma}
\prf
We need to construct a pair of homotopy equivalences $ r\colon B \to B^\prime$ and $l\colon B^\prime\to B$ that fit in a commuting diagram like
    \begin{center}
        \begin{tikzcd}[row sep=small]
            B \arrow[rr, transform canvas={yshift=.3ex}, "r"] \arrow[dr, swap, "\t"] & & B^\prime \arrow[ll, transform canvas={yshift=-.3ex}, "l"] \arrow[dl, "\t^\prime"]\\
            & \BO & 
        \end{tikzcd}
    \end{center}
This suggests that $r$ should be the lift of $\t$ along $\t^\prime$ and $l$ vice verca the lift of $\t^\prime$ along $\t$.
By the symmetry of the situation it suffices to detail the construction of $r$. 
%Let $B^{(i)}$ denote the $i$-skeleton of a CW-decomposition of $B$. 
%The $i$-th lifting obstruction for the skeletal lifting problem
%\begin{center}
    %\begin{tikzcd}
        %B^{(i+1)} \arrow[r, dashed, "r^{(i+1)}"] & B^\prime \arrow[d,"\theta^\prime"] \\
        %B^{(i)} \arrow[u,hook]\arrow[r, "\t"]\arrow[ur,"r^{(i)}"] & \BO
    %\end{tikzcd}
%\end{center}
%is a class in $H^{i+1}\big(\BO,\pi_i(\hofib (\theta^\prime))\big)$. 
%Since $\theta^\prime$ is $k$-coconnected, $\pi_i(\hofib (\theta^\prime)) = 0$ for $i \geq k$, and there arise no problems with lifting in this range.
%For $i< k$ any approach with obstruction theory is doomed, since one always needs $\pi_1(\BO)$ to act trivially on $\pi_i(B^\prime)$, a luxury that we can not hope to enjoy.
An approach with obstruction theory is doomed, since one needs $\pi_1(\BO)$ to act trivially on $\pi_i(\B^\prime)$ for this, a luxury that we cannot afford.
Therefore a more bare-bones argument is required.\\
It is a nifty change of perspective that does the trick:
Instead of extending the map $\theta$, in the following we view $B^\prime$ as a sort of subspace in $\BO$, and compress $\t$ down to this subspace.
The Map $\hat{\st}\colon M\to B$ is $k$-connected, so (after replacing $B$ with the mapping cylinder of $\hat{\st}$, which we may do up to equivalence of fibrations) we can view $(B,M)$ as a relative CW-complex made by only attaching cells of dimension $> k$. 
Similarly, since $\t^\prime$ is $k$-coconnected, we may view $\BO$ as being built from $B^\prime$ by attaching cells of dimension $\leq k$.
We now look at the map $(\t,\hat{\st}^\prime)\colon (B,M) \to (\BO,B^\prime)$ of relative CW-complexes. 
\begin{center}
    \begin{tikzcd}[row sep=tiny]
        B \arrow[r,"\t"]\arrow[ddr, dashed, "r"] & \BO\\
        \mathrm{Cyl}{(\hat{\st})} \arrow[draw=none]{u}[sloped,auto=false]{\whe} & \mathrm{Cyl}{(\t^\prime)} \arrow[draw=none]{u}[sloped,auto=false]{\whe}\\ 
        M \arrow[draw=none]{u}[sloped,auto=false]{\subseteq}\arrow[r,"\hat{\st}^\prime"] & B^\prime  \arrow[draw=none]{u}[sloped,auto=false]{\subseteq}
    \end{tikzcd}
\end{center}
And by what Hatcher calls the \emph{compression lemma} the dashed compression $r$ exists, if $\pi_i(\hofib(\t^\prime))$ vanishes in those degrees where we glued cells to $M$ to obain $B$, exactly what we have.
The map $r$ is homotopic to $\theta$ viewed as maps to $\mathrm{Cyl}(\theta^\prime)$, so it is $k$-coconnected.
Since the inclusion of $M$ into $\mathrm{Cyl}(\hat{\st})$ is $k$-connected and the diagram commutes, one easily also sees that $r$ must be $k$-connected.
Therefore $r$ is a weak homotopy equivalence, and by Whiteheads theorem even a homotopy equivalence.
\endprf
We have now established that the tangential $k$-type is indeed a well defined invariant of a manifold.
Let us compute the tangential $k$-type of the Sphere $S^d$ with $d>k$ first.
We have $\pi_i(S^d) = 0$ for $0<i<d$, so to admit a $k$-connected map $S^d \to B^k(S^d)$ we have to have $\pi_i(B^k(S^d)) = 0$ for $0<i\leq k$ aswell.
In colloquial terms we thus search for a space that looks like $\BO$ with killed lower $k$ homotopy groups. 
Such a space we find in the $k$-th stage of the Whitehead-tower of $\BO$. 
The lifting obstruction of the stable tangent bundle $\st\colon S^d \to \BO$ for the $i$-th stage is given by the homotopy class of a map $S^d \to K(\pi_i(\BO),i)$, or equivalently an element in $\pi_d(K(\pi_i(\BO),i))$.
Since for $i\leq k <d$ we have $\pi_d(K(\pi_i(\BO),i))= 0$, the stable tangent bundle lifts to a $k$-connected map $S^d\to \BO\langle k \rangle$.
By construction $\BO\langle k \rangle \to \BO$ is $k$-coconnected, thus the example is concluded: $B^k(S^d) = \BO\langle k\rangle$ for $k<d$.\\
The main result of this chapter will be the classification of all tangential $2$-types. 
To prepare ourselfes, we begin with the easier classification of tangential $1$-types, which is closely related to the earlier example of orientable vector bundles.
\begin{thesisprop}
    Let $M$ be a connected manifold with first Stiefel Whitney class $w_1 = w_1(TM)\in H^1(M;\zz)$. Then the tangential $1$-type of $M$ is ...
    \begin{itemize}[noitemsep]
        \item[...] $(\BO,\id_\BO)$ if $w_1 \neq 0$.
        \item[...] $(\BSO, \BSO \to \BO)$ if $w_1 = 0$.
    \end{itemize}
\end{thesisprop}
\prf
    Both $\id_\BO$ and $\BSO\to\BO$ are $1$-coconnected, so all there is to study is the connectivity of $\st\colon M\to \BO$.
    Either $\pi_1(\st)$ is surjective, then $\st$ is already $1$-connected and the tangential type is $\id_\BO$, or $\pi_1(\st)$ not surjective, implying $\im(\pi_1(\st)) = \{0\}$ since $\pi_1(\BO) \cong \zz$. 
    In the second case $\st$ then lifts to a map into the universal cover $\BSO \to \BO$, which is trivially $1$-connected.\\
    To connect the Stiefel Whitney class with the vanishing of the homeomorphism $\pi_1(\st)$ we first use the fact that $\pi_1(\BO)= \zz$ is abelian to descend via Hurewicz to the map $\st_\ast$ on $H_1$.
    Using Universal Coefficients we get 
    \begin{center}
        \begin{tikzcd}[row sep=small]
            0 = \Ext_\Z^1(\Z,\zz) \arrow[r, ""] & H^1(M;\zz) \arrow[r, "\cong"] & \Hom_\Z(H_1(M;\Z),\zz) \\ 
            0 = \Ext_\Z^1(\Z,\zz) \arrow[r, ""] & H^1(\BO ;\zz) \arrow[r, "\cong"] \arrow[u, "\st^\ast"] & \Hom_\Z(H_1(\BO ;\Z),\zz ) \arrow[u, "\Hom(\bullet{,}\,\zz)( \st_\ast )"] 
        \end{tikzcd}
    \end{center}
    and so $\pi_1(\st)$ is the zero map if and only if $\st^\ast$ is the zero map. But since $\st^\ast(w_1(\BO)) = w_1$, this depends on whether $w_1 = 0$.
\endprf
Following is a rather involved computation of the possible tangential $2$-types of connected manifolds. 
The difficulty in comparison to above arises from the fundamental group of $M$ playing a much more pronounced role when we demand $M\to B^2(M)$ to be $2$-connected.
\begin{thesisprop}\label{tttypes}
    Let $M$ be a connected manifold with Fundamental group $G = \pi_1(M)$ first and second Stiefel Whitney classes $w_i = w_i(TM) \in H^i(M;\zz)$. 
    Denote the Universal Cover of $M$ by $\widetilde{M}$ and the second Stiefel Whitney class of the Universal Cover by $\widetilde{w}_2 = w_2(T\widetilde{M}) \in H^2(\widetilde{M};\zz)$.
    Then we have six different families of tangential $2$-types, parametrized by $G$:
    \begin{enumerate}[noitemsep, label=\Roman*., labelindent=3cm, labelwidth=!]
        \item $ \widetilde{w}_2 \neq 0, w_1 \neq 0, w_2 \neq 0,\quad$ resulting in type $\BO \times_{\B \zz} \B G$,
        \item $ \widetilde{w}_2 \neq 0, w_1 = 0, w_2 \neq 0,\quad$ resulting in type $\BSO \times \B G$,
        \item[III. - VI.] $\widetilde{w}_2 = 0\quad\quad\quad\quad\quad\quad\quad\quad\,$ resulting in type $\B(G,w_1,w_2)$.
    \end{enumerate}
    For the defining fibrations to $\BO$ and the definition of $\B(G,w_1,w_2)$ consult the proof.
\end{thesisprop}
The keen eye immeadiately recognizes there are two cases missing, namely those with $w_2 = 0$ but $\widetilde{w}_2 \neq 0$. 
While these are perfectly fine cases in the abstract construction below, they of course never appear as a tangential $2$-type:
The covering map $p\colon \widetilde{M} \to M$ gives a bundle map $T\widetilde{M} \to T M $ and thus by naturality $\widetilde{w}_2 = p^\ast w_2$. 
Therefore $w_2 = 0$ implies $\widetilde{w}_2 = 0$.
Before diving into the proof of \ref{tttypes}, we will first climb to a slightly higher outpost of abstraction to unify the approaches instead of handeling all six cases separately.
\begin{defi}[Tangential datum]
    Let $G$ be a group. A tangential datum over $G$ is an equivalence class of triples $(G,w_1,w_2)$ where $w_i \in H^i(\B G ;\zz)$. Two such triples $(G,w_1,w_2)$ and $(G^\prime, w_1^\prime, w_2^\prime)$ are called equivalent if there exists an Isomorphism $\phi\colon G \to G^\prime$ with $\phi^\ast w_i^\prime = w_i$.
\end{defi}
A manifold $M$ gives rise to (parts of) a tangential datum as follows: 
To noones surprise, set $G = \pi_1(M)$.
We may think of the univeral cover as being classified by a $2$-connected Map $u\colon M \to \B G$.
Notice how by Hurewicz and Universal coefficients then also $u^\ast\colon H^1(\B G;\zz) \to H^1(M;\zz)$ is an isomorphism.
Therefore we have a unique class $w_1 \in H^1(\B G,\zz)$ with $u^\ast w_1 = w_1(M)$.\\
In the case that $w_2(\widetilde{M}) = p^\ast w_2(M) = 0$ we use the exact sequence 
\begin{equation*}\tag{$\heartsuit$}\label{heartseq}
    0 \to H^2(\B G ; \zz) \overset{u^\ast}{\to} H^2(M;\zz) \overset{p^\ast}{\to} H^2(\widetilde{M}) \cdots
\end{equation*}
to similarly obtain a unique class $w_2 \in H^1(\B G,\zz)$ with $u^\ast w_2 = w_2(M)$. 
Otherwise, if $w_2(\widetilde{M})\neq 0$, the statement and discussion of \ref{tttypes} suggest we will not need any further information than $w_1$, and the proof will underpin this.\\
The sequence $\heartsuit$ comes from a slightly involved argument, which \cite{kreck:sad} attributes to \cite{brown:cog}. 
For sake of accessibility, the argument shall be expanded here.
Up to replacing $M$ and $\widetilde{M}$ by weakly homotopy equivalent spaces, the classifying map of the universal cover $u$ gives rise to a fibration sequence
\begin{equation*}
    \widetilde{M} \fib M \overset{u}{\to} \B G.
\end{equation*}
From the Atiyah--Hirzebruch spectral sequence $E^{p,q}$ of this fibration with $\zz$ coefficients we observe (see \fref{fig:ahss}) the following facts, using constantly that $H^1(\widetilde{M},\zz) = 0$ and thus $E_2^{q,1} = 0$ for all $q\in\N$.
\begin{enumerate}[label=(\roman*)]
    \item The second cohomology of $\B G$ at $E_2^{2,0}$ is not hit by any nontrivial differentials, hence we have $H^2 (\B G; \zz) = E_2^{2,0} = E_\infty^{2,0}$.
    \item The second cohomology of $\widetilde{M}$ at $E_2^{0,2} = E_3^{0,2}$ admits a single nontrivial differential to $E_3^{3,0}$, which we call $d_3$. Then $E_\infty^{0,2} = E_4^{0,2} = \ker(d_3) \subset E_3^{0,2} = H^2(\widetilde{M},\zz)$.
    \item Since we have field coefficients, we compute the second cohomology of $M$ as $H^2(M;\zz) \cong E_\infty^{0,2} \oplus E_\infty^{1,1}\oplus E_\infty^{2,0} = \ker(d_3) \oplus H^2(\B G,\zz)$.
\end{enumerate}
\begin{figure}
    \centering
    \includegraphics[width=.48\textwidth]{img/tikz/e2.pdf}
    \includegraphics[width=.48\textwidth]{img/tikz/e_infty.pdf}
    \caption{The $E_2$ page (left) and $E_\infty$ page (right) in the Atiyah--Hirzebruch spectral sequence of $u$.}\label{fig:ahss}
\end{figure}
By including into and projecting from the direct sum in (iii) to its components, we obtain a sequence
\begin{equation*}
    0 \to H^2(\B G;\zz) = E_\infty^{2,0} \hookrightarrow H^2(M;\zz) \cong \ker(d_3) \oplus H^2(\B G;\zz) \to \ker(d_3) \hookrightarrow H^2(\widetilde{M};\zz).
\end{equation*}
This sequence is the desired sequence $\heartsuit$.
exactness at $H^2(\B G;\zz)$ is obvious, as it just amounts to the Injectivity of the inclusion Map.
Due to the monomorphism property of the last Map, the exactness of $\heartsuit$ at $H^2(M;\zz)$ is ruled by the evident exactness of the above sequence at the Direct Sum.\\
The only caveat left is the identification of the composite $H^2(M;\zz) \to \ker(d_3) \to H^2(\widetilde{M};\zz)$ with $p^\ast$.
This fact is a not something specific to this case, but rather a general fact of the Atiyah--Hirezebruch-sequence of a fibration.
A proof can be for example found as the proof of theorem 5.~9 in \cite{mc:guide}.
After this digression to the world of spectral sequence calculations, we are finally ready to prove \ref{tttypes}.\\
\prf
    We will digest the cases in the order they appear in the proposition.\\
    \textbf{I.} As a quick warmup, let us handle the first case $w_2(\widetilde{M}) \neq 0$ and $w_1(M) \neq 0$. 
    Begin by obtaining $(G,w_1)$ as illustrated above.
    Let $\overline{w}_1\in H^1(\BO; \zz)$ denote the first universal Stiefel--Whitney-class.
    Since $\B\zz$ is an Eilenberg--MacLane-space $K(\zz, 1)$ We may think of $\overline{w}_1$ as being represented by a map $\BO \to \B\zz$, which we name $\overline{w}_1$ by slight abuse of notation.
    The square
    \begin{center}
        \begin{tikzcd}
            M \arrow[r,"u"]\arrow[d,"\st"] & \B G \arrow[d,"w_1"] \\
            \BO \arrow[r,"\overline{w}_1"] & \B\zz 
        \end{tikzcd}
    \end{center}
    commutes, since $\st^\ast \overline{w}_1 = w_1(TM) = u^\ast w_1$ by definition.
    We therefore get a well defined map $\hat{\st}$ to the pullback which is a lift of $\st$ along $\theta$.
    \begin{center}
        \begin{tikzcd}
             & \BO\times_{\B\zz}\B G \arrow[r]\arrow[d,"\theta"] & \B G \arrow[d,"w_1"] \\
            M \arrow[r,"\st"] \arrow[dashed, ur,"\hat{\st}"] & \BO \arrow[r,"\overline{w}_1"] & \B\zz 
        \end{tikzcd}
    \end{center}
    Replacing $\B G$ with a weakly equivalent space, we may assume that $w_1$ is a fibration.
    Hence the pullback above is a homotopy pullback and we have $\pi_i(\hofib (\theta)) \cong \pi_i(\hofib (w_1))$.
    The latter vanishes for $i\geq 2$, as $\pi_i(\B\zz) \cong 0 \cong \pi_i(\B G)$ for $i\geq 2$ and we have the LES of $w_1$:
    \begin{equation*}
        \cdots \to \pi_{i+1}(\B\zz) \to \pi_i(\hofib (w_1)) \to \pi_i(\B G) \to \cdots
    \end{equation*}
    We conclude, that the map $\theta$ is indeed $2$-coconnected.\\
    Comparing the two long exact sequences of the vertical fibrations $\theta$ and $w_1$ we obtain
    \begin{center}
        \begin{tikzcd}[column sep=small]
            \pi_2(\BO) \arrow[r]\arrow[d] & \pi_1(\hofib (\theta)\arrow[r]\arrow[d,"\cong"] & \pi_1(\BO \times_{\B\zz} \B G) \arrow[r,"{\pi_1(\theta)}"]\arrow[d,"r"] & \pi_1(\BO) \arrow[r]\arrow[d,"{\pi_1(\overline{w}_1)}"] & \pi_0(\hofib(\theta)) \arrow[d,"\cong"] \\
            0 = \pi_2(\B\zz) \arrow[r] & \pi_1(\hofib (w_1)) \arrow[r] & \pi_1(\B G) \arrow[r,"{\pi_1(w_1)}"] & \pi_1(\B\zz) \arrow[r] & \pi_0(\hofib(w_1)).
        \end{tikzcd}
    \end{center}
    The homeomorphism $\pi_1(\overline{w}_1)$ is nontrivial and since $\pi_1(\BO) \cong \zz$ also an isomorphism. 
    The $5$-lemma thus gives the ratiocination that $\pi_1(\BO\times_{\B\zz} \B G) \cong G$.
    Since the isomorphism $\pi_1(u)$ factors over $\pi_1(\BO\times_{\B\zz}\B G)$ in 
    \begin{center}
        \begin{tikzcd}
            M \arrow[d,"\hat{\st}"] \arrow[dr, "u"] & \\
            \BO\times_{\B\zz}\B G \arrow[r] & \B G 
        \end{tikzcd}
    \end{center}
    we can conclude that $\pi_1(\hat{\st})$ is an isomorphism by the two--out--of--three-property.
    Note that $\pi_2(\BO\times_{\B\zz}\B G) \cong \pi_2(\BO) \cong \zz$ from consultation with the LES of the horizontal maps in the pullback diagram and $5$-lemma.
    Hence for showing that $\pi_2(\hat{\st})$ is an epimorphism, it is sufficient to show that $\pi_2(\st)$ is an epimorphism.
    Since $w_2(\widetilde{M}) \neq 0$, the classifying map of the stable tangent bundle of $\widetilde{M}$ is an epimorphism onto $\pi_2(\BO)$.
    It factors as $\st\circ p$, since $p$ is a local diffeomorphism, so therefore also $\st$ is an epimorphism on $\pi_2$, concluding the argument that $\hat{\st}$ is $2$-connected.\\ 
    \textbf{II.} In the case $w_2(\widetilde{M}) \neq 0$ and $w_1(M) = 0$, the composition of fibrations
    \begin{equation*}
        \theta\colon \B G \times \BSO \overset{\mathrm{pr}}{\to} \BSO \overset{t}{\to} \BO
    \end{equation*}
    is the tangential type of $M$, where $t\colon \BSO \to \BO$ is the universal covering map.
    Since $M$ is orientable, the stable tangent bundle of $M$ lifts along $t$ to a oriented bundle $\hat{\st} \colon M \to \BSO$.
    The lift of $\st$ along $\theta$ is now given by the tuple $(u,\hat{\st})$, where $u\colon M \to \B G$ as always denotes the classifying map of the universal cover $p\colon \widetilde{M} \to M$.\\
    The fiber $\B G$ of the projection $\mathrm{pr}$ is $2$-coconnected, so $\mathrm{pr}$ is $2$-coconnected.
    Similarly, also $t$ is as a covering map $2$-coconnected, and by composition therefore $\theta$ is $2$-coconnected.
    As we know, the map $u$ is an isomorphism on $\pi_1$, and since $\pi_1(\BSO) = 0$, this makes $(u,\hat{\st})$ an isomorphism on $\pi_1$.
    On the level of $\pi_2$, the complementary situation arises:\\
    With $\pi_2(\B G) = 0$ the homeomorphism $\pi_2((u,\hat{\st}))$ is completely determined by $\pi_2(\hat{\st})$.
    A similar idea as in case I now applies.
    An orientation of $\st$ fixes a compatible orientation of the stable tangent bundle of $\widetilde{M}$.
    We may thus think of the stable tangent bundle of the universal cover of $M$ as a map $\widetilde{\st}\colon\widetilde{M} \to \BSO$ that satisfies $\widetilde{\st} = \hat{\st}\circ p$.
    Again, $\pi_2(\widetilde{\st})$ may not be the trivial map, for else $w_2(\widetilde{M})$ was trivial.
    Since $\pi_2(\BSO) \cong \zz$, this immediately implies surjectivity of $\pi_2(\widetilde{\st})$.
    And again by the same argument as in case I we obtain surjectivity of $\pi_2(\hat{\st})$ as a consequence.\\
    \textbf{III.- VI.}
    In the case $w_2(\widetilde{M})= 0$ first obtain a tangential datum $(G,w_1,w_2)$ from $M$.
    We again want to think of $w_i\in H^i(\B G; \zz)$ as being represented by maps to the according Eilenberg--MacLane-space $w_i \colon \B G \to K(\zz,i)$.
    We also consider the universal Stiefel--Whitney-classes $\overline{w}_i\in H^i(\BO;\zz)$ as being represented by such maps.
    For readability we introduce the notation $\B^2\zz$ for $K(\zz ,2)$.
    Replace again $\B G$ by a weakly equivalent space such tha t $(w_1,w_2)\colon \B G \to \B \zz \times \B^2 \zz$ becomes a fibration.
    Then for the same reason as in the warmup case the square
    \begin{center}
        \begin{tikzcd}
            M \arrow[r,"u"]\arrow[d,"\st"] & \B G \arrow[d,"{(w_1,w_2)}"] \\
            \BO \arrow[r,"{(\overline{w}_1,\overline{w}_2)}"] & \B \zz \times \B^2 \zz
        \end{tikzcd}
    \end{center}
   commutes, and we get a well defined lift $\hat{\st}$ along the fibration $\theta$ in the (homotopy) pullback
    \begin{center}
        \begin{tikzcd}
            & \B (G,w_1,w_2) \arrow[r]\arrow[d,"\theta"] & \B G \arrow[d,"{(w_1,w_2)}"] \\
            M \arrow[r,"\st"] \arrow[dashed, ur,"\hat{\st}"] & \BO \arrow[r,"{(\overline{w}_1,\overline{w}_2)}"] & \B \zz \times \B^2 \zz
        \end{tikzcd}
    \end{center}
    Analogous to the first case we use the LES of the fibration $(w_1,w_2)$ together with the observation that $\pi_i(\B G) = 0$ for $i\geq 2$ and $\pi_i(\B\zz \times \B^2\zz) = 0$ for $i\geq 3$ to compute that for $i\geq 2$ we have $\pi_i(\hofib(\theta))\cong \pi_i(\hofib((w_1,w_2))) \cong 0 $.
    We thus establish that $\theta$ is $2$-coconnected.\\
    Turning our eye towars the analysis of $\hat{\st}$ we first notice, that again due to the commutativity of 
    \begin{center}
        \begin{tikzcd}
            M \arrow[d,"\hat{\st}"] \arrow[dr, "u"] & \\
            \B (G,w_1,w_2) \arrow[r] & \B G 
        \end{tikzcd}
    \end{center}
   we get injectivity of $\pi_1(\hat{\st})$ for free. 
   Writing down the $5$-lemma diagram from the vertical fibrations again and using that $\pi_i(\overline{w}_i)$ are isomorphisms, as in case I we see $\pi_1(\B(G,w_1,w_2)) \cong G$, so $\pi_1(\hat{\st})$ is an isomorphism.
   The only different part of the calculation compared to the first case concerns the surjectivity of $\pi_2(\hat{\st})$.
   It is trivially fulfilled since $\pi_2(\B(G,w_1,w_2)) = 0$ as one can infer with yet another invocation of the $5$-lemma for
    \begin{center}
        \begin{tikzcd}[column sep=small]
            0\arrow[r] & 0\arrow[r]\arrow[d,"\cong"] & \pi_2(\B(G, w_1,w_2)) \arrow[r,"{\pi_2(\theta)}"]\arrow[d] & \pi_2(\BO) \arrow[r]\arrow[d,"{(0,\pi_2(\overline{w}_2))}"] & \pi_1(\hofib(\theta)) \arrow[d,"\cong"] \\
            0\arrow[r] & 0 \arrow[r] & \pi_2(\B G) \arrow[r,"{\pi_2((w_1,w_2))}"] & \pi_2(\B\zz\times\B^2\zz) \arrow[r] & \pi_1(\hofib((w_1,w_2)))
        \end{tikzcd}
    \end{center}
    with noting $\pi_2(\B G) = 0$.
    This concludes the computation of the possible tangential $2$-types of manifolds.
\endprf
As a remark, let us more explicitly determine the space $\B(G,0,0)$. 
If $M$ has vanishing $w_1$ and $w_2$, the stable tangent bundle $\st$ lifts along the $2$-coconnected fibration $\BSpin \to\BO$ from the Whitehead-tower.
The universal cover together with this lift form a map $M\to \BSpin \times \B G$ which is $2$-connected by an argument analogous to the argument in the second case of the proof.
By uniqueness (\ref{ttunique}), we get that the tangential $2$-type $\B(G,0,0)$ is just $\theta\colon \BSpin\times\B G \overset{\mathrm{pr}}{\to} \BSpin \to \BO$.\\
As a first application of the proposition, let us compute the tangential $2$-type of the real projective space $\RP^m$ for $m>1$.
We remind ourselfes, that the total Stiefel--Whitney-class of $\RP^m$ is given by $w= (1+c)^{m+1}$, where $c$ is a nontrivial element of degree $1$ representing the tautological bundle, subject only to the obvious relation $c^{m+1} = 0$.
The binomial theorem hands us
\begin{equation*}
    w_1(\RP^m) = (m+1)c,\quad w_2(\RP^m) = \begin{pmatrix} m+1 \\ 2 \end{pmatrix}c^2
\end{equation*}
and we immediately recognize the well known fact that $\RP^m$ is orientable if and only if $m$ is odd.
The analysis of $w_2$ is mildly laborious.
First notice, that the parity of the binomial coefficient in this case only depends on the remainder of $m$ dividing by $4$, since generally 
\begin{equation*}\tag{$\diamondsuit$}
    \begin{pmatrix} i\\2 \end{pmatrix} \mod 2 \equiv \begin{cases} 0 & i \equiv 0,1 \mod 4 \\ 1 & i\equiv 2,3 \mod 4.\end{cases}
\end{equation*}
Combining this with our knowledge that the universal cover $S^m$ of $\RP^m$ is always spinnable, and the fact that $\pi_1(\RP^m) \cong \zz$, we have as possible tangential types of the real projective spaces
\begin{center}\vspace{.5\baselineskip}
    \begin{tabular}{c|c|c|c|c}\hline\hline
        $m \mod 4$ & $0$ & $1$ & $2$ & $3$\\\hline
        $w_1(\RP^m) $ & $c$ & $0$ & $c$ & $0$\\
        $w_2(\RP^m) $ & $0$ & $c^2$ & $c^2$ & $0$\\
        Type & $\B(\zz,w_1,0)$ & $\B(\zz,0,w_2)$ & $\B(\zz,w_1,w_2)$ & $\BSpin\times\B\zz$ \\\hline\hline
    \end{tabular}\vspace{.5\baselineskip}
\end{center}
We now conclude this chapter with a bigger list of examples for each tangential type family.\\
 \textbf{I.} The $\BO\times\B G$ type family is for example inhabited by manifolds like $\RP^m\times\CP^n$ for $n,m\neq 0$ even.
        With $\RP^m$ being nonorientable for even $m$, so is the product, thus $w_1\neq 0$.
        For $w_2$, consult the Whitney product theorem to compute
        \begin{equation*}
                w_2(\RP^m\times\CP^n) = w_2(\RP^m) + w_1(\RP^m)w_1(\CP^n) + w_2(\CP^n),
        \end{equation*}
        where we disregard in our notation the obvious pullbacks along the canonical projections which make above formula welldefined.
        The total Stiefel--Whitney-class of $\CP^n$ is given by $(1+d)^{n+1}$ where $d$ is a degree $2$ element subject only to the relation $d^{n+1}= 0$.
        The binomial theorem now computes
        \begin{equation*}
                w_2(\CP^n) = (n+1)d
        \end{equation*}
        So indeed the $w_2(\CP^n) \neq 0$ iff $n$ even.
        The universal cover of $\RP^m\times\CP^n$ is given by $S^m\times\CP^n$.
        Since all Stiefel--Whitney-classes of $S^m$ vanish, by the Whitney product theorem we have $w_2(S^m\times\CP^n) = w_2(\CP^n)$ which is nonzero for $n$ even.\\
\textbf{II.} In the $\BSO\times \B G$ type family of manifolds we have the even complex projective spaces $\CP^{2n}$, as they are orientable  and have nonvanishing $w_2$ by above. 
        They are furthermore simply connected, which one sees nicely by the LES of the fibration sequence $S^1 \fib S^{2n+1} \to \CP^n$.
        Hence they are their own universal cover, so their universal cover isnt spinnable aswell.\\
\textbf{III.} In the $\BSpin\times\B G \to \BO$ type family we have with $G$ trivial all manifolds that satisfy the condition in \ref{moregeomspin}, as well as all spheres $S^d$ with $d>2$ and the odd complex projective spaces $\CP^{2n+1}$.
        Examples of manifolds with nontrivial fundamental group in this type family are given by $S^1$ and, more generally, all Tori since they are parallelizable.
        Furthermore all orientable $3$-manifolds are parallelizable, see \emph{Satz 21} in \cite{stiefel:cc} for the historical proof of this marvellous fact, and thus also belong to this family.
        Lastly, also in dimensions $1$ and $2$, all orientable manifolds are spinnable as we have discussed.\\
\textbf{I.-VI.} The \hypertarget{doldmnf}{\emph{Dold-manifolds}} $P(m,n)$ with odd $n$ provide more excellent examples in the type familes $\B(G,w_1,w_2)$.
        Define $P(m,n) = S^m\times\CP^n /_\sim$ where $(x,z) \sim (-x,\overline{z})$, a manifold with universal (double) cover $S^m\times\CP^n$ for $m>1$.
        By example I, their universal cover is thus spin exactly when $n$ is odd. 
        The total Stiefel--Whitney class of $P(m,n)$ is given by $w = (1+c)^m(1+c+d)^{n+1}$, where $c,d$ are nontrivial elements of degrees $1$ and $2$ (originally calculated by \cite{dold:bord}).
        It is straightaway computed that we get $ w_1(P(m,n)) = (n + m + 1)c$  vanishes iff $n$ and $m$ do not have the same parity.
        For the second Stiefel--Whitney-class we obtain
        \begin{equation*}
                w_2(P(m,n)) = (n+1)d + \left(\begin{pmatrix} m\\ 2 \end{pmatrix} + \begin{pmatrix} n +1 \\ 2\end{pmatrix} + (n+1)m\right)c^2.
        \end{equation*}
        Now remembering again identity $\diamondsuit$, there are only eight cases to consider, since we already fixed $n$ to be odd.
        \begin{sidewaysfigure}
            \centering
            \includegraphics[width=\textwidth]{img/tikz/pmn.pdf}
            \caption{The tangential type of the Dold manifolds $P(m,n)$ for varying $n>0$ and $m>1$. The symbol $/$ stands for a vanishing Stiefel--Whitney-class.}\label{fig:pmn}
        \end{sidewaysfigure}
        All these cases can found in the table in \fref{fig:pmn}, together with the cases for even $n$ giving tangential types I and II. \\
